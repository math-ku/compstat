% Options for packages loaded elsewhere
% Options for packages loaded elsewhere
\PassOptionsToPackage{unicode}{hyperref}
\PassOptionsToPackage{hyphens}{url}
\PassOptionsToPackage{dvipsnames,svgnames,x11names}{xcolor}
%
\documentclass[
  ignorenonframetext,
  aspectratio=1610,
  onlytextwidth]{beamer}
\newif\ifbibliography
\usepackage{pgfpages}
\setbeamertemplate{caption}[numbered]
\setbeamertemplate{caption label separator}{: }
\setbeamercolor{caption name}{fg=normal text.fg}
\beamertemplatenavigationsymbolsempty
% remove section numbering
\setbeamertemplate{part page}{
  \centering
  \begin{beamercolorbox}[sep=16pt,center]{part title}
    \usebeamerfont{part title}\insertpart\par
  \end{beamercolorbox}
}
\setbeamertemplate{section page}{
  \centering
  \begin{beamercolorbox}[sep=12pt,center]{section title}
    \usebeamerfont{section title}\insertsection\par
  \end{beamercolorbox}
}
\setbeamertemplate{subsection page}{
  \centering
  \begin{beamercolorbox}[sep=8pt,center]{subsection title}
    \usebeamerfont{subsection title}\insertsubsection\par
  \end{beamercolorbox}
}
% Prevent slide breaks in the middle of a paragraph
\widowpenalties 1 10000
\raggedbottom
\AtBeginPart{
  \frame{\partpage}
}
\AtBeginSection{
  \ifbibliography
  \else
    \frame{\sectionpage}
  \fi
}
\AtBeginSubsection{
  \frame{\subsectionpage}
}
\usepackage{iftex}
\ifPDFTeX
  \usepackage[T1]{fontenc}
  \usepackage[utf8]{inputenc}
  \usepackage{textcomp} % provide euro and other symbols
\else % if luatex or xetex
  \usepackage{unicode-math} % this also loads fontspec
  \defaultfontfeatures{Scale=MatchLowercase}
  \defaultfontfeatures[\rmfamily]{Ligatures=TeX,Scale=1}
\fi
\usepackage{lmodern}

\usetheme[]{moloch}
\usecolortheme[]{moloch-tomorrow}
\ifPDFTeX\else
  % xetex/luatex font selection
\fi
% Use upquote if available, for straight quotes in verbatim environments
\IfFileExists{upquote.sty}{\usepackage{upquote}}{}
\IfFileExists{microtype.sty}{% use microtype if available
  \usepackage[]{microtype}
  \UseMicrotypeSet[protrusion]{basicmath} % disable protrusion for tt fonts
}{}

\usepackage{color}
\usepackage{fancyvrb}
\newcommand{\VerbBar}{|}
\newcommand{\VERB}{\Verb[commandchars=\\\{\}]}
\DefineVerbatimEnvironment{Highlighting}{Verbatim}{commandchars=\\\{\}}
% Add ',fontsize=\small' for more characters per line
\usepackage{framed}
\definecolor{shadecolor}{RGB}{248,248,248}
\newenvironment{Shaded}{\begin{snugshade}}{\end{snugshade}}
\newcommand{\AlertTok}[1]{\textcolor[rgb]{0.94,0.16,0.16}{#1}}
\newcommand{\AnnotationTok}[1]{\textcolor[rgb]{0.56,0.35,0.01}{\textbf{\textit{#1}}}}
\newcommand{\AttributeTok}[1]{\textcolor[rgb]{0.13,0.29,0.53}{#1}}
\newcommand{\BaseNTok}[1]{\textcolor[rgb]{0.00,0.00,0.81}{#1}}
\newcommand{\BuiltInTok}[1]{#1}
\newcommand{\CharTok}[1]{\textcolor[rgb]{0.31,0.60,0.02}{#1}}
\newcommand{\CommentTok}[1]{\textcolor[rgb]{0.56,0.35,0.01}{\textit{#1}}}
\newcommand{\CommentVarTok}[1]{\textcolor[rgb]{0.56,0.35,0.01}{\textbf{\textit{#1}}}}
\newcommand{\ConstantTok}[1]{\textcolor[rgb]{0.56,0.35,0.01}{#1}}
\newcommand{\ControlFlowTok}[1]{\textcolor[rgb]{0.13,0.29,0.53}{\textbf{#1}}}
\newcommand{\DataTypeTok}[1]{\textcolor[rgb]{0.13,0.29,0.53}{#1}}
\newcommand{\DecValTok}[1]{\textcolor[rgb]{0.00,0.00,0.81}{#1}}
\newcommand{\DocumentationTok}[1]{\textcolor[rgb]{0.56,0.35,0.01}{\textbf{\textit{#1}}}}
\newcommand{\ErrorTok}[1]{\textcolor[rgb]{0.64,0.00,0.00}{\textbf{#1}}}
\newcommand{\ExtensionTok}[1]{#1}
\newcommand{\FloatTok}[1]{\textcolor[rgb]{0.00,0.00,0.81}{#1}}
\newcommand{\FunctionTok}[1]{\textcolor[rgb]{0.13,0.29,0.53}{\textbf{#1}}}
\newcommand{\ImportTok}[1]{#1}
\newcommand{\InformationTok}[1]{\textcolor[rgb]{0.56,0.35,0.01}{\textbf{\textit{#1}}}}
\newcommand{\KeywordTok}[1]{\textcolor[rgb]{0.13,0.29,0.53}{\textbf{#1}}}
\newcommand{\NormalTok}[1]{#1}
\newcommand{\OperatorTok}[1]{\textcolor[rgb]{0.81,0.36,0.00}{\textbf{#1}}}
\newcommand{\OtherTok}[1]{\textcolor[rgb]{0.56,0.35,0.01}{#1}}
\newcommand{\PreprocessorTok}[1]{\textcolor[rgb]{0.56,0.35,0.01}{\textit{#1}}}
\newcommand{\RegionMarkerTok}[1]{#1}
\newcommand{\SpecialCharTok}[1]{\textcolor[rgb]{0.81,0.36,0.00}{\textbf{#1}}}
\newcommand{\SpecialStringTok}[1]{\textcolor[rgb]{0.31,0.60,0.02}{#1}}
\newcommand{\StringTok}[1]{\textcolor[rgb]{0.31,0.60,0.02}{#1}}
\newcommand{\VariableTok}[1]{\textcolor[rgb]{0.00,0.00,0.00}{#1}}
\newcommand{\VerbatimStringTok}[1]{\textcolor[rgb]{0.31,0.60,0.02}{#1}}
\newcommand{\WarningTok}[1]{\textcolor[rgb]{0.56,0.35,0.01}{\textbf{\textit{#1}}}}

\usepackage{longtable,booktabs,array}
\usepackage{calc} % for calculating minipage widths
\usepackage{caption}
% Make caption package work with longtable
\makeatletter
\def\fnum@table{\tablename~\thetable}
\makeatother
\usepackage{graphicx}
\makeatletter
\newsavebox\pandoc@box
\newcommand*\pandocbounded[1]{% scales image to fit in text height/width
  \sbox\pandoc@box{#1}%
  \Gscale@div\@tempa{\textheight}{\dimexpr\ht\pandoc@box+\dp\pandoc@box\relax}%
  \Gscale@div\@tempb{\linewidth}{\wd\pandoc@box}%
  \ifdim\@tempb\p@<\@tempa\p@\let\@tempa\@tempb\fi% select the smaller of both
  \ifdim\@tempa\p@<\p@\scalebox{\@tempa}{\usebox\pandoc@box}%
  \else\usebox{\pandoc@box}%
  \fi%
}
% Set default figure placement to htbp
\def\fps@figure{htbp}
\makeatother





\setlength{\emergencystretch}{3em} % prevent overfull lines

\providecommand{\tightlist}{}



 


\usepackage[export]{adjustbox}
\usepackage{algorithm2e}
\usepackage{dirtree}
\setbeamertemplate{page number in head/foot}[appendixframenumber]
\setbeameroption{show notes}
\usepackage{fontsetup}
\makeatletter
\@ifpackageloaded{caption}{}{\usepackage{caption}}
\AtBeginDocument{%
\ifdefined\contentsname
  \renewcommand*\contentsname{Table of contents}
\else
  \newcommand\contentsname{Table of contents}
\fi
\ifdefined\listfigurename
  \renewcommand*\listfigurename{List of Figures}
\else
  \newcommand\listfigurename{List of Figures}
\fi
\ifdefined\listtablename
  \renewcommand*\listtablename{List of Tables}
\else
  \newcommand\listtablename{List of Tables}
\fi
\ifdefined\figurename
  \renewcommand*\figurename{Figure}
\else
  \newcommand\figurename{Figure}
\fi
\ifdefined\tablename
  \renewcommand*\tablename{Table}
\else
  \newcommand\tablename{Table}
\fi
}
\@ifpackageloaded{float}{}{\usepackage{float}}
\floatstyle{ruled}
\@ifundefined{c@chapter}{\newfloat{codelisting}{h}{lop}}{\newfloat{codelisting}{h}{lop}[chapter]}
\floatname{codelisting}{Listing}
\newcommand*\listoflistings{\listof{codelisting}{List of Listings}}
\makeatother
\makeatletter
\makeatother
\makeatletter
\@ifpackageloaded{caption}{}{\usepackage{caption}}
\@ifpackageloaded{subcaption}{}{\usepackage{subcaption}}
\makeatother
\makeatletter
\@ifpackageloaded{tikz}{}{\usepackage{tikz}}
\makeatother
        \newcommand*\circled[1]{\tikz[baseline=(char.base)]{
          \node[shape=circle,draw,inner sep=1pt] (char) {{\scriptsize#1}};}}  
                  

\usepackage{bookmark}
\IfFileExists{xurl.sty}{\usepackage{xurl}}{} % add URL line breaks if available
\urlstyle{same}
\hypersetup{
  pdftitle={Introduction},
  pdfauthor={Johan Larsson},
  colorlinks=true,
  linkcolor={black},
  filecolor={Maroon},
  citecolor={Blue},
  urlcolor={DodgerBlue3},
  pdfcreator={LaTeX via pandoc}}


\title{Introduction}
\subtitle{Computational Statistics}
\author{Johan Larsson}
\date{August 26, 2025}
\institute{Department of Mathematical Sciences, University of
Copenhagen}
\titlegraphic{
  \includegraphics[right]{../images/ucph-horizontal-left.pdf}}

\begin{document}
\frame{\titlepage}


\begin{frame}{What is Computational Statistics?}
\phantomsection\label{what-is-computational-statistics}
It is a broad field, where meaning depends on context.

\pause

One definition is that it is \textbf{the use of computational methods to
solve statistical problems}, for instance

\pause

\begin{itemize}
\tightlist
\item
  simulation,
\item
  optimization,
\item
  numerical integration,
\item
  data analysis, and
\item
  visualization.
\end{itemize}
\end{frame}

\begin{frame}{A Running Example}
\phantomsection\label{a-running-example}
\begin{columns}[T]
\begin{column}{0.47\linewidth}
Let's try to get a bit of flavor of what we will be doing in the course.

\bigskip

\pause

Throughout the course we will use a data set of amino acid angles,
\(\Phi\) and \(\Psi\), from protein structures.
\end{column}

\begin{column}{0.47\linewidth}
\begin{figure}[H]

{\centering \includegraphics[width=1\linewidth,height=\textheight,keepaspectratio]{../images/PhiPsi_creative.jpg}

}

\caption{Amino Acid Angles}

\end{figure}%
\end{column}
\end{columns}
\end{frame}

\begin{frame}[fragile]{Histograms}
\phantomsection\label{histograms}
A simple way to analyze the distributions of the angles \(\Phi\) and
\(\Psi\) is the \textbf{histogram}.

\bigskip

\pause

\begin{columns}[T]
\begin{column}{0.47\linewidth}
\begin{Shaded}
\begin{Highlighting}[]
\FunctionTok{ggplot}\NormalTok{(phipsi, }\FunctionTok{aes}\NormalTok{(}\AttributeTok{x =}\NormalTok{ phi)) }\SpecialCharTok{+}
  \FunctionTok{geom\_histogram}\NormalTok{() }\SpecialCharTok{+}
  \FunctionTok{geom\_rug}\NormalTok{(}\AttributeTok{alpha =} \FloatTok{0.5}\NormalTok{) }\SpecialCharTok{+}
  \FunctionTok{labs}\NormalTok{(}
    \AttributeTok{x =} \FunctionTok{expression}\NormalTok{(Phi),}
    \AttributeTok{y =} \StringTok{"Density"}
\NormalTok{  )}
\end{Highlighting}
\end{Shaded}
\end{column}

\begin{column}{0.47\linewidth}
\pandocbounded{\includegraphics[keepaspectratio]{lecture1_files/figure-beamer/hist1-show-1.pdf}}
\end{column}
\end{columns}
\end{frame}

\begin{frame}[fragile]{Density Estimation}
\phantomsection\label{density-estimation}
Histograms are not very smooth. If we allow ourselves to make stronger
assumptions, we can get a smoother estimate of the distribution, using
\textbf{kernel density estimation} (KDE).

\bigskip

\pause

\begin{columns}[T]
\begin{column}{0.47\linewidth}
\begin{Shaded}
\begin{Highlighting}[]
\FunctionTok{ggplot}\NormalTok{(phipsi, }\FunctionTok{aes}\NormalTok{(}\AttributeTok{x =}\NormalTok{ phi)) }\SpecialCharTok{+}
  \FunctionTok{geom\_density}\NormalTok{() }\SpecialCharTok{+}
  \FunctionTok{geom\_rug}\NormalTok{(}\AttributeTok{alpha =} \FloatTok{0.5}\NormalTok{) }\SpecialCharTok{+}
  \FunctionTok{labs}\NormalTok{(}
    \AttributeTok{x =} \FunctionTok{expression}\NormalTok{(Phi),}
    \AttributeTok{y =} \StringTok{"Density"}
\NormalTok{  )}
\end{Highlighting}
\end{Shaded}
\end{column}

\begin{column}{0.47\linewidth}
\end{column}
\end{columns}

\pause

But how is this KDE actually computed? Doing this efficiently is a
\textbf{computational statistics} problem.
\end{frame}

\begin{frame}{Statistical Topics of the Course}
\phantomsection\label{statistical-topics-of-the-course}
The course can be broken down into a number of \textbf{statistical} and
\textbf{computational} topics.

\pause

There are three statistical topics in the course:

\pause

\begin{block}{Smoothing}
\phantomsection\label{smoothing}
We will learn how to compute efficient kernel density estimates and
scatterplot smoothers.

\pause
\end{block}

\begin{block}{Simulation}
\phantomsection\label{simulation}
We will learn to efficiently simulate from probability distributions
using inversion, rejection, and importance sampling.

\pause
\end{block}

\begin{block}{Optimization}
\phantomsection\label{optimization}
We will learn to solve optimization problems that arise in statistics,
for instance in maximum likelihood estimation (MLE), using the EM
algorithm and gradient-based optimization.
\end{block}
\end{frame}

\begin{frame}{Computational Topics of the Course}
\phantomsection\label{computational-topics-of-the-course}
\begin{block}{Implementation}
\phantomsection\label{implementation}
We will learn how to implement statistical methods in R, using
object-oriented programming and functional programming.

\pause
\end{block}

\begin{block}{Correctness}
\phantomsection\label{correctness}
We will learn how to ensure that our code is correct, using testing and
debugging.

\pause
\end{block}

\begin{block}{Efficiency}
\phantomsection\label{efficiency}
We will learn how measure performance and find bottlenecks in our code
using profiling and benchmarking, and how to optimize it.
\end{block}
\end{frame}

\begin{frame}{Teaching Staff}
\phantomsection\label{teaching-staff}
\begin{columns}[T]
\begin{column}{0.47\linewidth}
\begin{block}{Instructor}
\phantomsection\label{instructor}
Johan Larsson, postdoctoral researcher

\begin{figure}[H]

{\centering \includegraphics[width=0.5\linewidth,height=\textheight,keepaspectratio]{../images/johan.jpg}

}

\caption{Johan}

\end{figure}%

\begin{block}{Contact}
\phantomsection\label{contact}
Use Absalon for course-related questions and email (see Absalon) for
personal matters.
\end{block}
\end{block}
\end{column}

\pause

\begin{column}{0.47\linewidth}
\begin{block}{Teaching Assistant}
\phantomsection\label{teaching-assistant}
Jinyang Liu, PhD student in machine learning

\begin{figure}[H]

{\centering \includegraphics[width=0.5\linewidth,height=\textheight,keepaspectratio]{../images/jinyang.jpg}

}

\caption{Jin}

\end{figure}%
\end{block}
\end{column}
\end{columns}
\end{frame}

\begin{frame}{Assignments}
\phantomsection\label{assignments}
Four assignments make up the bulk of the course work.

\pause

\bigskip

For each assignment, there are two alternatives (A and B). You will pick
one.

\pause

\bigskip

Each assignment is tied to a particular \textbf{topic}:

\begin{enumerate}
\tightlist
\item
  Smoothing
\item
  Univariate simulation
\item
  The EM algorithm
\item
  Stochastic optimization
\end{enumerate}
\end{frame}

\begin{frame}{Presentations}
\phantomsection\label{presentations}
There will be four presentation sessions (week 3, 4, 6,
7)\footnote<.->[frame]{Not counting the potato harvesting week when you
  are off.}

\pause

\bigskip

You will divide into groups of 2-3 students and present your solution to
one of the assignments during one of the sessions.

\pause

\bigskip

You will register for groups and assignments in
\href{https://absalon.ku/dk}{Absalon}.

\pause

\bigskip

Presentation is compulsory but not graded. We expect solutions to be
work in progress.

\pause
\end{frame}

\begin{frame}{Oral Examination}
\phantomsection\label{oral-examination}
The main examination is an oral exam based on your assignments.

\bigskip

\pause

You will prepare four presentations, one for each assignment you picked.

\bigskip

\pause

At the exam, you will present one of these at random.
\end{frame}

\begin{frame}{Schedule}
\phantomsection\label{schedule}
\begin{columns}[T]
\begin{column}{0.47\linewidth}
\begin{block}{Lectures}
\phantomsection\label{lectures}
\begin{itemize}
\tightlist
\item
  Tuesdays and Thursdays, 10:15--12:00 (Johan)
\end{itemize}

\pause
\end{block}

\begin{block}{Exercise Sessions}
\phantomsection\label{exercise-sessions}
\begin{itemize}
\tightlist
\item
  Thursdays, 08:15--10:00 (Jinyang)
\end{itemize}

\pause
\end{block}

\begin{block}{Presentations}
\phantomsection\label{presentations-1}
\begin{itemize}
\tightlist
\item
  Thursdays, 13:15--15:00 (Johan)
\item
  Only weeks 3, 4, 6, and 7
\end{itemize}
\end{block}
\end{column}

\pause

\begin{column}{0.47\linewidth}
\begin{block}{Examination}
\phantomsection\label{examination}
\begin{itemize}
\tightlist
\item
  November 6-8 (8.15-17.30, tentative)
\item
  Rooms to be announced
\end{itemize}
\end{block}
\end{column}
\end{columns}
\end{frame}

\begin{frame}[fragile]{Course Literature}
\phantomsection\label{course-literature}
\begin{block}{Computational Statistics with R}
\phantomsection\label{computational-statistics-with-r}
Main textbook for the course, written by Niels Richard Hansen.

\begin{itemize}
\tightlist
\item
  Available online at \url{https://cswr.nrhstat.org/}
\item
  Not yet complete, but we only use parts that are.
\item
  \href{https://github.com/nielsrhansen/CSwR/tree/master/CSwR_package}{Companion
  package}: install with
  \texttt{pak::pak("github::nielsrhansen/CSwR/CSwR\_package")}.
\end{itemize}

\pause
\end{block}

\begin{block}{Advanced R}
\phantomsection\label{advanced-r}
Auxiliary textbook, written by Hadley Wickham.

\begin{itemize}
\tightlist
\item
  Available online at \url{https://adv-r.hadley.nz/}
\item
  Covers more advanced R programming topics.
\item
  We will use selected chapters.
\end{itemize}
\end{block}
\end{frame}

\begin{frame}{Online Resources}
\phantomsection\label{online-resources}
\begin{columns}[T]
\begin{column}{0.7\linewidth}
\begin{block}{Absalon}
\phantomsection\label{absalon}
Main source for information and communication about the course. Accessed
at \href{https://absalon.ku.dk/}{absalon.ku.dk}.

\pause
\end{block}

\begin{block}{CompStat Web Page}
\phantomsection\label{compstat-web-page}
Course content will be uploaded to
\href{https://math-ku.github.io/compstat/}{github.io/math-ku/compstat}.

\medskip

This is where you will a detailed schedule of the course, slides, and
the assignments.
\end{block}
\end{column}

\begin{column}{0.25\linewidth}
\begin{figure}[H]

{\centering \includegraphics[width=1\linewidth,height=\textheight,keepaspectratio]{../images/absalon.jpg}

}

\caption{Absalon}

\end{figure}%
\end{column}
\end{columns}
\end{frame}

\begin{frame}{Generative AI}
\phantomsection\label{generative-ai}
Generative AI (e.g.~ChatGPT, Copilot, Bard, etc.) are powerful tools.

\bigskip

\pause

You are allowed to use them in this course, but with some caveats:

\begin{itemize}
\tightlist
\item
  You must understand the results.
\item
  You must acknowledge their use in your assignments and how you used
  them.
\end{itemize}

\pause

\bigskip

\begin{columns}[T]
\begin{column}{0.47\linewidth}
\begin{block}{Copilot}
\phantomsection\label{copilot}
Access to GitHub Copilot is available for free to students, via
\href{https://education.github.com}{GitHub Education}.
\end{block}
\end{column}

\begin{column}{0.47\linewidth}
\includegraphics[width=1\linewidth,height=\textheight,keepaspectratio]{../assets/images/github-education.png}
\end{column}
\end{columns}
\end{frame}

\section{Programming in R}\label{programming-in-r}

\begin{frame}[fragile]{Prerequisite R Knowledge}
\phantomsection\label{prerequisite-r-knowledge}
We expect knowledge of

\begin{itemize}
\tightlist
\item
  data structures (vectors, lists, data frames),
\item
  control structures (loops, if-then-else),
\item
  function calling,
\item
  interactive and script usage (\texttt{source}) of R.
\end{itemize}

\pause

All of this is covered in chapters 1-5 of
\href{https://adv-r.hadley.nz/}{Advanced R}.

\medskip

But you \textbf{do not} need to be an experienced programmer.
\end{frame}

\section{Functions}\label{functions}

\begin{frame}[fragile]{Functions in R}
\phantomsection\label{functions-in-r}
\begin{itemize}
\tightlist
\item
  Everything that happens in R is the result of a function call. Even
  \texttt{+}, \texttt{{[}} and \texttt{\textless{}-} are functions.
\item
  An R function takes a number of \emph{arguments}, and when a function
  call is evaluated it computes a \emph{return value}.
\item
  Functions can return any R object, including functions!
\item
  Implementations of R functions are collected into source files, which
  can be organized into packages.
\item
  The order of your functions in the script does not matter.
\end{itemize}
\end{frame}

\begin{frame}[fragile]{Components of a Function}
\phantomsection\label{components-of-a-function}
\begin{columns}[T]
\begin{column}{0.47\linewidth}
\begin{block}{Arguments}
\phantomsection\label{arguments}
\begin{Shaded}
\begin{Highlighting}[]
\NormalTok{f }\OtherTok{\textless{}{-}} \ControlFlowTok{function}\NormalTok{(x, y) \{}
\NormalTok{  x }\SpecialCharTok{+}\NormalTok{ y}
\NormalTok{\}}
\end{Highlighting}
\end{Shaded}

\pause

\begin{Shaded}
\begin{Highlighting}[]
\FunctionTok{formals}\NormalTok{(f)}
\end{Highlighting}
\end{Shaded}

\begin{verbatim}
$x


$y
\end{verbatim}
\end{block}
\end{column}

\pause

\begin{column}{0.47\linewidth}
\begin{block}{Body}
\phantomsection\label{body}
\begin{Shaded}
\begin{Highlighting}[]
\FunctionTok{body}\NormalTok{(f)}
\end{Highlighting}
\end{Shaded}

\begin{verbatim}
{
    x + y
}
\end{verbatim}

\pause
\end{block}

\begin{block}{Environment}
\phantomsection\label{environment}
\begin{Shaded}
\begin{Highlighting}[]
\FunctionTok{environment}\NormalTok{(f)}
\end{Highlighting}
\end{Shaded}

\begin{verbatim}
<environment: R_GlobalEnv>
\end{verbatim}
\end{block}
\end{column}
\end{columns}
\end{frame}

\begin{frame}[fragile]{Naming Functions}
\phantomsection\label{naming-functions}
\begin{quote}
There are only two hard things in Computer Science: cache invalidation
and naming things.

\medskip

\hfill\emph{--Phil Karlton}
\end{quote}

\pause

\begin{columns}[T]
\begin{column}{0.47\linewidth}
\begin{block}{Favor Descriptive Names}
\phantomsection\label{favor-descriptive-names}
Begin to see if you can use a \emph{verb}. Better long and descriptive
than short and cryptic.

\pause
\end{block}

\begin{block}{Honor Common Conventions}
\phantomsection\label{honor-common-conventions}
Avoid \texttt{.} in names; it is used for \textbf{methods} (upcoming).

\pause
\end{block}
\end{column}

\begin{column}{0.47\linewidth}
\begin{block}{Use a Consistent Style}
\phantomsection\label{use-a-consistent-style}
\begin{itemize}
\tightlist
\item
  \texttt{lowercase}
\item
  \texttt{snake\_case} (tidyverse)
\item
  \texttt{camelCase}
\item
  \texttt{UpperCamelCase}
\end{itemize}

\pause
\end{block}

\begin{block}{Namespace Clashes}
\phantomsection\label{namespace-clashes}
\begin{itemize}
\tightlist
\item
  Avoid names of existing functions.
\end{itemize}
\end{block}
\end{column}
\end{columns}
\end{frame}

\begin{frame}[fragile]{Example: Counting Zeros}
\phantomsection\label{example-counting-zeros}
Count data is often modeled using a Poisson distribution. R can simulate
count data using the function \texttt{rpois()}.

\begin{Shaded}
\begin{Highlighting}[]
\FunctionTok{rpois}\NormalTok{(}\DecValTok{10}\NormalTok{, }\DecValTok{2}\NormalTok{) }\CommentTok{\# n = 10 variables from a Poisson(2) distribution}
\end{Highlighting}
\end{Shaded}

\begin{verbatim}
 [1] 0 2 2 2 4 2 0 1 2 2
\end{verbatim}

\pause

There are two zeros in this sequence.

\pause

Let's write a function that counts the number of zeros: checks for zero
inflation.
\end{frame}

\begin{frame}[fragile]{A First Attempt}
\phantomsection\label{a-first-attempt}
\begin{Shaded}
\begin{Highlighting}[]
\NormalTok{count\_zeros }\OtherTok{\textless{}{-}} \ControlFlowTok{function}\NormalTok{(x) \{}
\NormalTok{  n\_zeros }\OtherTok{\textless{}{-}} \DecValTok{0}
  \ControlFlowTok{for}\NormalTok{ (i }\ControlFlowTok{in} \DecValTok{1}\SpecialCharTok{:}\FunctionTok{length}\NormalTok{(x)) \{}
    \ControlFlowTok{if}\NormalTok{ (x[i] }\SpecialCharTok{==} \DecValTok{0}\NormalTok{) \{}
\NormalTok{      n\_zeros }\OtherTok{\textless{}{-}}\NormalTok{ n\_zeros }\SpecialCharTok{+} \DecValTok{1}
\NormalTok{    \}}
\NormalTok{  \}}
\NormalTok{  n\_zeros}
\NormalTok{\}}
\end{Highlighting}
\end{Shaded}

\pause

\begin{Shaded}
\begin{Highlighting}[]
\FunctionTok{count\_zeros}\NormalTok{(}\FunctionTok{c}\NormalTok{(}\DecValTok{3}\NormalTok{, }\DecValTok{2}\NormalTok{, }\DecValTok{0}\NormalTok{))}
\end{Highlighting}
\end{Shaded}

\begin{verbatim}
[1] 1
\end{verbatim}

\pause

\begin{Shaded}
\begin{Highlighting}[]
\FunctionTok{count\_zeros}\NormalTok{(}\FunctionTok{c}\NormalTok{(}\DecValTok{0}\NormalTok{, }\DecValTok{0}\NormalTok{, }\DecValTok{0}\NormalTok{))}
\end{Highlighting}
\end{Shaded}

\begin{verbatim}
[1] 3
\end{verbatim}
\end{frame}

\begin{frame}{Testing}
\phantomsection\label{testing}
It is critical to ensure that your code does what you think it does, and
that it continues to do so as you modify it.

\pause

\bigskip

Testing is the process of writing code that checks that your code is
correct.

\pause

\bigskip

Some people even think that the \textbf{first thing} you should do is to
write a test.

\pause

\bigskip

As you modify your code, your tests will catch these
\textbf{regressions} for you.
\end{frame}

\begin{frame}[fragile]{testthat}
\phantomsection\label{testthat}
\href{https://testthat.r-lib.org/}{testthat} is the most popular testing
framework for R.

\pause

\begin{Shaded}
\begin{Highlighting}[]
\CommentTok{\# In file tests/test\_count\_zeros.R}
\FunctionTok{test\_that}\NormalTok{(}\StringTok{"count\_zeros work on various input"}\NormalTok{, \{}
  \FunctionTok{expect\_equal}\NormalTok{(}\FunctionTok{count\_zeros}\NormalTok{(}\FunctionTok{c}\NormalTok{(}\DecValTok{0}\NormalTok{, }\DecValTok{0}\NormalTok{, }\FloatTok{1e{-}9}\NormalTok{, }\DecValTok{25}\NormalTok{)), }\DecValTok{2}\NormalTok{)}
  \FunctionTok{expect\_equal}\NormalTok{(}\FunctionTok{count\_zeros}\NormalTok{(}\FunctionTok{c}\NormalTok{(}\SpecialCharTok{{-}}\DecValTok{0}\NormalTok{, }\FloatTok{1.1}\NormalTok{, }\SpecialCharTok{{-}}\DecValTok{2}\NormalTok{)), }\DecValTok{1}\NormalTok{)}
  \FunctionTok{expect\_equal}\NormalTok{(}\FunctionTok{count\_zeros}\NormalTok{(}\FunctionTok{c}\NormalTok{()), }\DecValTok{0}\NormalTok{)}
\NormalTok{\})}
\end{Highlighting}
\end{Shaded}

\begin{Shaded}
\begin{Highlighting}[]
\NormalTok{testthat}\SpecialCharTok{::}\FunctionTok{test\_dir}\NormalTok{(}\StringTok{"tests"}\NormalTok{)}
\end{Highlighting}
\end{Shaded}

\begin{verbatim}
Error in `testthat::test_dir()`:
! No test files found
\end{verbatim}
\end{frame}

\begin{frame}[fragile]{A Second Attempt}
\phantomsection\label{a-second-attempt}
\begin{Shaded}
\begin{Highlighting}[]
\NormalTok{count\_zeros }\OtherTok{\textless{}{-}} \ControlFlowTok{function}\NormalTok{(x) \{}
\NormalTok{  n\_zeros }\OtherTok{\textless{}{-}} \DecValTok{0}
  \ControlFlowTok{for}\NormalTok{ (i }\ControlFlowTok{in} \FunctionTok{seq\_along}\NormalTok{(x)) \{}
    \ControlFlowTok{if}\NormalTok{ (x[i] }\SpecialCharTok{==} \DecValTok{0}\NormalTok{) \{}
\NormalTok{      n\_zeros }\OtherTok{\textless{}{-}}\NormalTok{ n\_zeros }\SpecialCharTok{+} \DecValTok{1}
\NormalTok{    \}}
\NormalTok{  \}}
\NormalTok{  n\_zeros}
\NormalTok{\}}
\end{Highlighting}
\end{Shaded}

\begin{Shaded}
\begin{Highlighting}[]
\NormalTok{testthat}\SpecialCharTok{::}\FunctionTok{test\_dir}\NormalTok{(}\StringTok{"tests"}\NormalTok{)}
\end{Highlighting}
\end{Shaded}

\begin{verbatim}
v | F W  S  OK | Context

/ |          0 | count_zeros                                                    
v |          3 | count_zeros

== Results =====================================================================
[ FAIL 0 | WARN 0 | SKIP 0 | PASS 3 ]
\end{verbatim}
\end{frame}

\begin{frame}{Debugging}
\phantomsection\label{debugging}
\begin{itemize}
\tightlist
\item
  Sometimes hard to identify the offending piece of code.
\item
  Helpful to use a debugging tool. R studio comes with a helpful
  interface for this.
\item
  We will talk more about debugging in week 5.
\end{itemize}
\end{frame}

\begin{frame}[fragile]{Functional Programming}
\phantomsection\label{functional-programming}
In functional programming, functions are \textbf{first-class citizens}:
they can be passed as arguments to other functions, returned as values
from functions, and assigned to variables.

\bigskip

\pause

This allows for a high degree of abstraction and code reuse, for
instance through the use of the \texttt{apply} family of functions.

\pause

Let's write our own apply function.

\bigskip

\begin{Shaded}
\begin{Highlighting}[]
\NormalTok{our\_apply }\OtherTok{\textless{}{-}} \ControlFlowTok{function}\NormalTok{(x, fun) \{}
\NormalTok{  val }\OtherTok{\textless{}{-}} \FunctionTok{numeric}\NormalTok{(}\FunctionTok{length}\NormalTok{(x))}
  \ControlFlowTok{for}\NormalTok{ (i }\ControlFlowTok{in} \FunctionTok{seq\_along}\NormalTok{(x)) \{}
\NormalTok{    val[i] }\OtherTok{\textless{}{-}} \FunctionTok{fun}\NormalTok{(x[[i]])}
\NormalTok{  \}}
\NormalTok{  val}
\NormalTok{\}}
\end{Highlighting}
\end{Shaded}
\end{frame}

\begin{frame}[fragile]{Testing Our Apply Function}
\phantomsection\label{testing-our-apply-function}
\begin{Shaded}
\begin{Highlighting}[]
\FunctionTok{sapply}\NormalTok{(}\DecValTok{1}\SpecialCharTok{:}\DecValTok{10}\NormalTok{, exp)}
\end{Highlighting}
\end{Shaded}

\begin{verbatim}
 [1]     2.718282     7.389056    20.085537    54.598150   148.413159
 [6]   403.428793  1096.633158  2980.957987  8103.083928 22026.465795
\end{verbatim}

\pause

\begin{Shaded}
\begin{Highlighting}[]
\FunctionTok{our\_apply}\NormalTok{(}\DecValTok{1}\SpecialCharTok{:}\DecValTok{10}\NormalTok{, exp)}
\end{Highlighting}
\end{Shaded}

\begin{verbatim}
 [1]     2.718282     7.389056    20.085537    54.598150   148.413159
 [6]   403.428793  1096.633158  2980.957987  8103.083928 22026.465795
\end{verbatim}

\pause

\begin{block}{Assumptions}
\phantomsection\label{assumptions}
\texttt{x} is a list, \texttt{fun()} takes a single argument, and
\texttt{fun()} returns a numeric.

\pause
\end{block}
\end{frame}

\begin{frame}[fragile]{What if \texttt{fun()} Needs Additional
Arguments?}
\phantomsection\label{what-if-fun-needs-additional-arguments}
Then we get an error:

\begin{Shaded}
\begin{Highlighting}[]
\FunctionTok{our\_apply}\NormalTok{(}\DecValTok{1}\SpecialCharTok{:}\DecValTok{10}\NormalTok{, rpois)}
\end{Highlighting}
\end{Shaded}

\begin{verbatim}
Error in fun(x[[i]]): argument "lambda" is missing, with no default
\end{verbatim}

\pause

\begin{block}{Anonymous Functions}
\phantomsection\label{anonymous-functions}
We can use an anonymous function to pass additional arguments to
\texttt{fun()}.

\begin{Shaded}
\begin{Highlighting}[]
\FunctionTok{our\_apply}\NormalTok{(}
  \DecValTok{1}\SpecialCharTok{:}\DecValTok{10}\NormalTok{,}
  \ControlFlowTok{function}\NormalTok{(lambda) }\FunctionTok{rpois}\NormalTok{(}\DecValTok{1}\NormalTok{, }\AttributeTok{lambda =} \FloatTok{0.9}\NormalTok{)}
\NormalTok{)}
\end{Highlighting}
\end{Shaded}

\begin{verbatim}
 [1] 0 0 1 2 0 2 3 1 1 0
\end{verbatim}
\end{block}
\end{frame}

\begin{frame}[fragile]{\texttt{...}}
\phantomsection\label{section}
More general functionality can be achieved wit the \texttt{...}
(ellipsis) argument, which passes arguments forward.

\pause

\phantomsection\label{annotated-cell-14}%
\begin{Shaded}
\begin{Highlighting}[]
\NormalTok{our\_apply }\OtherTok{\textless{}{-}} \ControlFlowTok{function}\NormalTok{(x, fun, ...) \{ }\hspace*{\fill}\NormalTok{\circled{1}}
\NormalTok{  val }\OtherTok{\textless{}{-}} \FunctionTok{numeric}\NormalTok{(}\FunctionTok{length}\NormalTok{(x))}
  \ControlFlowTok{for}\NormalTok{ (i }\ControlFlowTok{in} \FunctionTok{seq\_along}\NormalTok{(x)) \{}
\NormalTok{    val[i] }\OtherTok{\textless{}{-}} \FunctionTok{fun}\NormalTok{(x[[i]], ...) }\hspace*{\fill}\NormalTok{\circled{2}}
\NormalTok{  \}}
\NormalTok{  val}
\NormalTok{\}}
\end{Highlighting}
\end{Shaded}

\begin{description}
\tightlist
\item[\circled{1}]
\texttt{...} in the argument list of \texttt{our\_apply()} collects
additional arguments.
\item[\circled{2}]
\texttt{...} in the call to \texttt{fun()} passes these additional
arguments to \texttt{fun()}.
\end{description}

\pause

\begin{Shaded}
\begin{Highlighting}[]
\FunctionTok{our\_apply}\NormalTok{(}\DecValTok{1}\SpecialCharTok{:}\DecValTok{10}\NormalTok{, rpois, }\AttributeTok{n =} \DecValTok{1}\NormalTok{)}
\end{Highlighting}
\end{Shaded}

\begin{verbatim}
 [1]  0  1  4  3  7  6  8 16  8 12
\end{verbatim}
\end{frame}

\section{Benchmarking}\label{benchmarking}

\begin{frame}[fragile]{R Is Slow \ldots{}}
\phantomsection\label{r-is-slow}
\ldots{} when used like a low-level language.

\begin{itemize}
\tightlist
\item
  R is an \textbf{interpreted} (as opposed to \emph{compiled}) language.
\item
  It was written mainly for specifying statistical models (not for
  developing new numerical methods).
\item
  It is suitable for high-level programming where most low-level
  computations are implemented in a compiled language
  (e.g.~\texttt{lm()} and \texttt{qr()}.)
\item
  It is also quite old.
\end{itemize}
\end{frame}

\begin{frame}[fragile]{R Is Fast \ldots{}}
\phantomsection\label{r-is-fast}
\ldots{} when most computations are carried out by calls to compiled
code.

\begin{Shaded}
\begin{Highlighting}[]
\NormalTok{x }\OtherTok{\textless{}{-}} \FunctionTok{rnorm}\NormalTok{(}\FloatTok{1e4}\NormalTok{)}
\NormalTok{bench\_res }\OtherTok{\textless{}{-}}\NormalTok{ bench}\SpecialCharTok{::}\FunctionTok{mark}\NormalTok{(}
  \AttributeTok{loop =}\NormalTok{ \{}
\NormalTok{    y }\OtherTok{\textless{}{-}} \FunctionTok{numeric}\NormalTok{(}\FunctionTok{length}\NormalTok{(x))}
    \ControlFlowTok{for}\NormalTok{ (i }\ControlFlowTok{in} \FunctionTok{seq\_along}\NormalTok{(x)) \{}
\NormalTok{      y[i] }\OtherTok{\textless{}{-}} \DecValTok{10} \SpecialCharTok{*}\NormalTok{ x[i]}
\NormalTok{    \}}
\NormalTok{    y}
\NormalTok{  \},}
  \AttributeTok{vectorized =} \DecValTok{10} \SpecialCharTok{*}\NormalTok{ x}
\NormalTok{)}
\end{Highlighting}
\end{Shaded}
\end{frame}

\begin{frame}[fragile]{Plot Benchmark Results}
\phantomsection\label{plot-benchmark-results}
\begin{Shaded}
\begin{Highlighting}[]
\FunctionTok{autoplot}\NormalTok{(bench\_res)}
\end{Highlighting}
\end{Shaded}

\begin{figure}[H]

{\centering \pandocbounded{\includegraphics[keepaspectratio]{lecture1_files/figure-beamer/bench-plot-1.pdf}}

}

\caption{Benchmark results for a loop vs.~a vectorized computation}

\end{figure}%
\end{frame}

\begin{frame}[fragile]{Vectorization}
\phantomsection\label{vectorization}
Vectorization is the process of rewriting code to use vectorized
operations, which operate on entire vectors at once, instead of using
loops to operate on individual elements.

\pause

The term is somewhat misleading, since it does not necessarily involve
vector processors.

\pause

\begin{block}{Example: Counting Zeros, Vectorized}
\phantomsection\label{example-counting-zeros-vectorized}
\begin{Shaded}
\begin{Highlighting}[]
\NormalTok{count\_zeros\_vec }\OtherTok{\textless{}{-}} \ControlFlowTok{function}\NormalTok{(x) \{}
  \FunctionTok{sum}\NormalTok{(x }\SpecialCharTok{==} \DecValTok{0}\NormalTok{)}
\NormalTok{\}}
\end{Highlighting}
\end{Shaded}

\pause

\begin{itemize}
\tightlist
\item
  \texttt{x\ ==\ 0} checks if each entry of \texttt{x} is 0 and returns
  a vector of logicals.
\item
  \texttt{sum()} computes and returns the sum of all elements in a
  vector. Logicals are coerced to integers.
\item
  In this case the vectorized implementation is cohesive and clear.
\item
  The vectorized computations are performed by compiled code
  (C/C++/Fortran), which run faster than pure R code.
\item
  Writing vectorized code requires a larger knowledge of R functions.
\end{itemize}
\end{block}
\end{frame}

\begin{frame}[fragile]{Beware of Loops in Disguise}
\phantomsection\label{beware-of-loops-in-disguise}
Just because you ran a function, it does not mean that it is vectorized.

\pause

\begin{Shaded}
\begin{Highlighting}[]
\NormalTok{bench}\SpecialCharTok{::}\FunctionTok{mark}\NormalTok{(}
  \FunctionTok{sapply}\NormalTok{(x, }\ControlFlowTok{function}\NormalTok{(x\_i) }\DecValTok{10} \SpecialCharTok{*}\NormalTok{ x\_i),}
  \DecValTok{10} \SpecialCharTok{*}\NormalTok{ x}
\NormalTok{) }\SpecialCharTok{|\textgreater{}}
  \FunctionTok{plot}\NormalTok{()}
\end{Highlighting}
\end{Shaded}

\pandocbounded{\includegraphics[keepaspectratio]{lecture1_files/figure-beamer/unnamed-chunk-18-1.pdf}}
\end{frame}

\begin{frame}{Development Cycle Sketch}
\phantomsection\label{development-cycle-sketch}
\begin{itemize}
\tightlist
\item
  Is there a good-enough existing implemention for your problem? If yes,
  then you are done.
\item
  If not, implement a solution and test it. Does it solve your problem
  sufficiently well? If yes, then you're done.
\item
  If not, then profile (next week!), benchmark, and debug (week 5). Then
  refactor and optimize.
\end{itemize}

\pause

\begin{block}{The Root of All Evil}
\phantomsection\label{the-root-of-all-evil}
\medskip

\begin{quote}
We \emph{should} forget about small efficiencies, say about 97\% of the
time: premature optimization is the root of all evil. Yet we should not
pass up our opportunities in that critical 3\%.

\emph{---Donald Knuth}
\end{quote}
\end{block}
\end{frame}

\begin{frame}[fragile]{Example: Density Estimation}
\phantomsection\label{example-density-estimation}
\begin{Shaded}
\begin{Highlighting}[]
\NormalTok{kern\_dens }\OtherTok{\textless{}{-}} \ControlFlowTok{function}\NormalTok{(x, h, }\AttributeTok{m =} \DecValTok{512}\NormalTok{) \{}
\NormalTok{  rg }\OtherTok{\textless{}{-}} \FunctionTok{range}\NormalTok{(x)}
\NormalTok{  xx }\OtherTok{\textless{}{-}} \FunctionTok{seq}\NormalTok{(rg[}\DecValTok{1}\NormalTok{] }\SpecialCharTok{{-}} \DecValTok{3} \SpecialCharTok{*}\NormalTok{ h, rg[}\DecValTok{2}\NormalTok{] }\SpecialCharTok{+} \DecValTok{3} \SpecialCharTok{*}\NormalTok{ h, }\AttributeTok{length.out =}\NormalTok{ m)}
\NormalTok{  y }\OtherTok{\textless{}{-}} \FunctionTok{numeric}\NormalTok{(m)}

  \ControlFlowTok{for}\NormalTok{ (i }\ControlFlowTok{in} \FunctionTok{seq\_along}\NormalTok{(xx)) \{}
    \ControlFlowTok{for}\NormalTok{ (j }\ControlFlowTok{in} \FunctionTok{seq\_along}\NormalTok{(x)) \{}
\NormalTok{      y[i] }\OtherTok{\textless{}{-}}\NormalTok{ y[i] }\SpecialCharTok{+} \FunctionTok{exp}\NormalTok{(}\SpecialCharTok{{-}}\NormalTok{(xx[i] }\SpecialCharTok{{-}}\NormalTok{ x[j])}\SpecialCharTok{\^{}}\DecValTok{2} \SpecialCharTok{/}\NormalTok{ (}\DecValTok{2} \SpecialCharTok{*}\NormalTok{ h}\SpecialCharTok{\^{}}\DecValTok{2}\NormalTok{))}
\NormalTok{    \}}
\NormalTok{  \}}

\NormalTok{  y }\OtherTok{\textless{}{-}}\NormalTok{ y }\SpecialCharTok{/}\NormalTok{ (}\FunctionTok{sqrt}\NormalTok{(}\DecValTok{2} \SpecialCharTok{*}\NormalTok{ pi) }\SpecialCharTok{*}\NormalTok{ h }\SpecialCharTok{*} \FunctionTok{length}\NormalTok{(x))}

  \FunctionTok{list}\NormalTok{(}\AttributeTok{x =}\NormalTok{ xx, }\AttributeTok{y =}\NormalTok{ y)}
\NormalTok{\}}
\end{Highlighting}
\end{Shaded}
\end{frame}

\begin{frame}[fragile]{Vectorizing Our Density Estimator}
\phantomsection\label{vectorizing-our-density-estimator}
\begin{Shaded}
\begin{Highlighting}[]
\NormalTok{kern\_dens\_vec }\OtherTok{\textless{}{-}} \ControlFlowTok{function}\NormalTok{(x, h, }\AttributeTok{m =} \DecValTok{512}\NormalTok{) \{}
\NormalTok{  rg }\OtherTok{\textless{}{-}} \FunctionTok{range}\NormalTok{(x)}
\NormalTok{  xx }\OtherTok{\textless{}{-}} \FunctionTok{seq}\NormalTok{(rg[}\DecValTok{1}\NormalTok{] }\SpecialCharTok{{-}} \DecValTok{3} \SpecialCharTok{*}\NormalTok{ h, rg[}\DecValTok{2}\NormalTok{] }\SpecialCharTok{+} \DecValTok{3} \SpecialCharTok{*}\NormalTok{ h, }\AttributeTok{length.out =}\NormalTok{ m)}
\NormalTok{  y }\OtherTok{\textless{}{-}} \FunctionTok{numeric}\NormalTok{(m)}
\NormalTok{  const }\OtherTok{\textless{}{-}}\NormalTok{ (}\FunctionTok{sqrt}\NormalTok{(}\DecValTok{2} \SpecialCharTok{*}\NormalTok{ pi) }\SpecialCharTok{*}\NormalTok{ h }\SpecialCharTok{*} \FunctionTok{length}\NormalTok{(x))}

  \ControlFlowTok{for}\NormalTok{ (i }\ControlFlowTok{in} \FunctionTok{seq\_along}\NormalTok{(xx)) \{}
\NormalTok{    y[i] }\OtherTok{\textless{}{-}} \FunctionTok{sum}\NormalTok{(}\FunctionTok{exp}\NormalTok{(}\SpecialCharTok{{-}}\NormalTok{(xx[i] }\SpecialCharTok{{-}}\NormalTok{ x)}\SpecialCharTok{\^{}}\DecValTok{2} \SpecialCharTok{/}\NormalTok{ (}\DecValTok{2} \SpecialCharTok{*}\NormalTok{ h}\SpecialCharTok{\^{}}\DecValTok{2}\NormalTok{))) }\SpecialCharTok{/}\NormalTok{ const}
\NormalTok{  \}}

  \FunctionTok{list}\NormalTok{(}\AttributeTok{x =}\NormalTok{ xx, }\AttributeTok{y =}\NormalTok{ y)}
\NormalTok{\}}
\end{Highlighting}
\end{Shaded}
\end{frame}

\begin{frame}[fragile]{Benchmarking}
\phantomsection\label{benchmarking-1}
\begin{Shaded}
\begin{Highlighting}[]
\NormalTok{kern\_bench }\OtherTok{\textless{}{-}}\NormalTok{ bench}\SpecialCharTok{::}\FunctionTok{mark}\NormalTok{(}
  \FunctionTok{kern\_dens}\NormalTok{(phipsi}\SpecialCharTok{$}\NormalTok{psi, }\FloatTok{0.2}\NormalTok{),}
  \FunctionTok{kern\_dens\_vec}\NormalTok{(phipsi}\SpecialCharTok{$}\NormalTok{psi, }\FloatTok{0.2}\NormalTok{)}
\NormalTok{)}
\end{Highlighting}
\end{Shaded}
\end{frame}

\begin{frame}[fragile]{Plot Benchmark Results}
\phantomsection\label{plot-benchmark-results-1}
\begin{Shaded}
\begin{Highlighting}[]
\FunctionTok{plot}\NormalTok{(kern\_bench)}
\end{Highlighting}
\end{Shaded}

\pandocbounded{\includegraphics[keepaspectratio]{lecture1_files/figure-beamer/kern-bench-autoplot-1.pdf}}
\end{frame}

\begin{frame}[fragile]{Parameterized Benchmarking}
\phantomsection\label{parameterized-benchmarking}
\begin{Shaded}
\begin{Highlighting}[]
\NormalTok{kern\_benchmarks }\OtherTok{\textless{}{-}}\NormalTok{ bench}\SpecialCharTok{::}\FunctionTok{press}\NormalTok{(}
  \AttributeTok{n =} \DecValTok{2}\SpecialCharTok{\^{}}\NormalTok{(}\DecValTok{6}\SpecialCharTok{:}\DecValTok{9}\NormalTok{),}
  \AttributeTok{m =} \DecValTok{2}\SpecialCharTok{\^{}}\NormalTok{(}\DecValTok{5}\SpecialCharTok{:}\DecValTok{11}\NormalTok{),}
\NormalTok{  \{}
\NormalTok{    bench}\SpecialCharTok{::}\FunctionTok{mark}\NormalTok{(}
      \AttributeTok{loop =} \FunctionTok{kern\_dens}\NormalTok{(x[}\DecValTok{1}\SpecialCharTok{:}\NormalTok{n], }\AttributeTok{h =} \FloatTok{0.2}\NormalTok{, }\AttributeTok{m =}\NormalTok{ m),}
      \AttributeTok{vec =} \FunctionTok{kern\_dens\_vec}\NormalTok{(x[}\DecValTok{1}\SpecialCharTok{:}\NormalTok{n], }\AttributeTok{h =} \FloatTok{0.2}\NormalTok{, }\AttributeTok{m =}\NormalTok{ m)}
\NormalTok{    )}
\NormalTok{  \}}
\NormalTok{)}
\end{Highlighting}
\end{Shaded}
\end{frame}

\begin{frame}[fragile]{Plotting Results}
\phantomsection\label{plotting-results}
\begin{Shaded}
\begin{Highlighting}[]
\FunctionTok{library}\NormalTok{(tidyverse)}
\FunctionTok{mutate}\NormalTok{(kern\_benchmarks, }\AttributeTok{expression =} \FunctionTok{as.character}\NormalTok{(expression)) }\SpecialCharTok{|\textgreater{}}
  \FunctionTok{ggplot}\NormalTok{(}\FunctionTok{aes}\NormalTok{(m, median, }\AttributeTok{color =}\NormalTok{ expression)) }\SpecialCharTok{+}
  \FunctionTok{geom\_point}\NormalTok{() }\SpecialCharTok{+}
  \FunctionTok{geom\_line}\NormalTok{() }\SpecialCharTok{+}
  \FunctionTok{facet\_grid}\NormalTok{(}\AttributeTok{cols =} \FunctionTok{vars}\NormalTok{(n))}
\end{Highlighting}
\end{Shaded}

\pandocbounded{\includegraphics[keepaspectratio]{lecture1_files/figure-beamer/kern-bench-fig-1.pdf}}
\end{frame}

\begin{frame}{Getting Help with R}
\phantomsection\label{getting-help-with-r}
\begin{block}{Google It}
\phantomsection\label{google-it}
Especially good for error messages.

\pause
\end{block}

\begin{block}{Generative AI}
\phantomsection\label{generative-ai-1}
\begin{itemize}
\tightlist
\item
  Also great for error messages and debugging
\item
  \emph{Caution}: You need to understand the results, especially when
  you ask it to create something for you.
\end{itemize}

\pause
\end{block}

\begin{block}{Absalon Discussion Forum}
\phantomsection\label{absalon-discussion-forum}
Use the fact that there are twenty other people in the course with
exactly the same problem.
\end{block}
\end{frame}

\section{Exercises}\label{exercises}

\begin{frame}[fragile]{Exercises}
\begin{block}{Exercise 1}
\phantomsection\label{exercise-1}
Can you list three ways to access element \texttt{a} in this list?

\begin{Shaded}
\begin{Highlighting}[]
\NormalTok{l }\OtherTok{\textless{}{-}} \FunctionTok{list}\NormalTok{(}\AttributeTok{a =} \DecValTok{1}\NormalTok{, }\AttributeTok{b =} \DecValTok{2}\NormalTok{)}
\end{Highlighting}
\end{Shaded}

\pause
\end{block}

\begin{block}{Exercise 2}
\phantomsection\label{exercise-2}
Write a for loop that prints ``even'' if the loop variable is even,
``odd'' if the loop variable is odd, and exits if is larger than 10.
\end{block}
\end{frame}




\end{document}
