\PassOptionsToPackage{unicode,pdfusetitle}{hyperref}
\PassOptionsToPackage{hyphens}{url}
\PassOptionsToPackage{dvipsnames,svgnames,x11names}{xcolor}

\documentclass[aspectratio=1610,onlytextwidth]{beamer}

\usetheme{moloch}
\usefonttheme{professionalfonts}
\setbeamertemplate{page number in head/foot}[appendixframenumber]

\usepackage{dirtree}

\usepackage{listings}
\usepackage{lstautogobble}
\lstset{
  autogobble,
  language=R,
  showspaces=false,
  morekeywords={TRUE, FALSE},
  showtabs=false,
  breaklines=true,
  tabsize=2,
  keywordstyle=\color{SteelBlue4},
  stringstyle=\color{RedViolet},
  deletekeywords={data,frame,length,as,character},
  basicstyle=\ttfamily
}

\usepackage{lmodern}
\usepackage{amssymb,amsmath,mathtools,amsthm}
\usepackage[T1]{fontenc}
\usepackage{textcomp}

% \usepackage{minted}

\usepackage{upquote} % straight quotes in verbatim environments
\usepackage{microtype}
\UseMicrotypeSet[protrusion]{basicmath} % disable protrusion for tt fonts

\usepackage{xcolor}
\usepackage{xurl} % add URL line breaks if available
\usepackage{bookmark}
\usepackage{hyperref}

\usepackage{tikz}

\hypersetup{%
  colorlinks = true,
  linkcolor  = mLightGreen,
  filecolor  = mLightGreen,
  citecolor  = mLightGreen,
  urlcolor   = mLightGreen
}

% animations
\usepackage{xmpmulti}

%% subfigures
% \usepackage{subcaption}

% algorithms
\usepackage[ruled,vlined]{algorithm2e}
\resetcounteronoverlays{algocf}

\usepackage{booktabs}

\date{\today}
\titlegraphic{\hfill\includegraphics[width=4cm]{images/ucph-horizontal-right.pdf}\vspace{1cm}}

% bibliography
\usepackage[style=authoryear]{biblatex}
\addbibresource{lecture14.bib}

% title block
\title{Variations on Stochastic Gradient Descent}
\subtitle{Computational Statistics}
\author{Johan Larsson}
\institute{Department of Mathematical Sciences, University of Copenhagen}

% operators
\DeclareMathOperator*{\argmax}{arg\,max}

% macros
\newcommand{\pkg}[1]{\textsf{#1}}
\renewcommand{\vec}{\vectorsym}
\newcommand{\mat}{\matrixsym}
\newcommand{\du}{\mathrm{d}}


\begin{document}

\maketitle

% \begin{frame}[c]
%   \frametitle{Overview}
%
%   \tableofcontents
% \end{frame}
%

\begin{frame}[c]
  \frametitle{Last Time}

\end{frame}

\begin{frame}[c]
  \frametitle{Today}

  \begin{block}{Distributing and Organizing Code}

    \begin{itemize}
      \item Reproducibility
      \item R packages
    \end{itemize}

  \end{block}

  \pause

  \begin{block}{Course Summary}
    What did we actually do?
  \end{block}

  \pause

  \begin{block}{Oral Examination Prep}
    What to think of during examination
  \end{block}
\end{frame}

\section{Organizing Code as an R Package}

\begin{frame}[c]
  \frametitle{Organizing Code}

  \begin{columns}[T]
    \begin{column}{0.45\textwidth}
      \begin{block}{Components}
        \begin{itemize}
          \item Code for experiments
          \item Source code for functions (which we should be able to reuse)
          \item Tests
          \item Rcpp code
          \item Data
        \end{itemize}
      \end{block}

      There is a plethora of ways to organize this. Which one to choose?
    \end{column}

    \pause

    \begin{column}{0.45\textwidth}
      \begin{block}{R Package}
        One way is to make an R package, which helps in many ways:
        \begin{itemize}
          \item Easy to connect to C++ code through Rcpp.
          \item Built-in support for automatic testing
          \item Documentation
          \item Declare dependencies (other packages, R version)
        \end{itemize}
      \end{block}
    \end{column}
  \end{columns}
\end{frame}

\begin{frame}[c,fragile]
  \frametitle{R Packages}

  \begin{columns}
    \begin{column}{0.45\textwidth}
      Different approaches, but we will follow \textbf{R Packages}~\parencite{wickhamPackagesOrganizeTest2023},
      which is based around the \textbf{devtools} package.
    \end{column}
    \begin{column}{0.45\textwidth}
      \begin{figure}[htpb]
        \centering
        \frame{\includegraphics[width = 0.7\textwidth]{images/rpkgs-cover-2e-small.png}}
        \caption{%
          R Packages
        }
      \end{figure}%
    \end{column}
  \end{columns}
\end{frame}

\begin{frame}[c,fragile]
  \frametitle{Devtools}

  Meta-package for various helpers that
  aid in developing R packages (and projects).

  First off, install and load \textbf{devtools}:
  \begin{lstlisting}
        install.packages("devtools")
        library(devtools)
      \end{lstlisting}

  \begin{columns}
    \begin{column}{0.45\textwidth}

      This loads other packages that will be useful
      for setting up your package, most importantly the \textbf{usethis}
      package.
    \end{column}
    \begin{column}{0.45\textwidth}
      \begin{figure}[htpb]
        \centering
        \includegraphics[width=2.5cm]{images/usethis-logo.png}%
        \includegraphics[width=2.5cm]{images/devtools.pdf}
      \end{figure}
    \end{column}
  \end{columns}
\end{frame}

\begin{frame}[c]
  \frametitle{A Toy Example}

  \begin{block}{Rosenbrock Package}
    Let's build a simple package that solves the Rosenbrock optimization problem, i.e.
    find
    \[
      x^* = \operatorname{arg\,min}\left((a - x_1)^2 + b(x_2 - x_1^2)^2\right).
    \]
  \end{block}

  \bigskip\pause

  \begin{block}{What We Will Learn}
    \begin{itemize}[<+->]
      \item Adding R functions to our package
      \item Interfacing with Rcpp
      \item Testing our code
      \item Adding dependencies to other packages
      \item Licensing our package
    \end{itemize}
  \end{block}
\end{frame}

\begin{frame}[c,fragile]
  \frametitle{A First Package}
  \begin{block}{Create It}
    Call
    \begin{lstlisting}
          usethis::create_package("rosenbrock")
        \end{lstlisting}
    or use \texttt{File > New Project > New Directory > R Package using devtools} in R Studio.
  \end{block}

  \pause\bigskip

  \begin{columns}[T]
    \begin{column}{0.45\textwidth}
      This gives you a \alert{minimal} package:

      \medskip

      \dirtree{%
        .1 rosenbrock/.
        .2 R/.
        .2 DESCRIPTION.
        .2 NAMESPACE.
      }

      \medskip\pause

      You may also have \texttt{.Rbuildignore} and \texttt{.rosenbrock.Rproj}
      depending on how you created the package.
    \end{column}

    \pause

    \begin{column}{0.45\textwidth}
      \begin{block}{Install It}
        Open up the package in your editor (R Studio\footnote{In which case it
          should alread be opened.}).
        \begin{lstlisting}
          devtools::install()
        \end{lstlisting}

        \bigskip

        Voila, you have made an R package!
      \end{block}
    \end{column}
  \end{columns}

\end{frame}

\begin{frame}[standout]
  Thank you!
\end{frame}

\begin{frame}[c,fragile]
  \frametitle{R Code}

  \begin{block}{\texttt{.R/}}
    \begin{itemize}
      \item All R code should live in \texttt{.R}-files in \texttt{R/}.
      \item These files should (almost) always contain \alert{only} functions.
      \item Many ways to organize your files: one function per file, all functions
            of a certain S3 class in one file etc.
    \end{itemize}
  \end{block}

  \pause\bigskip

  Let's create a first file: \texttt{R/objective.R}. Use \lstinline{usethis::use_r("objective")} and insert this:

  \begin{lstlisting}
    objective <- function(x, a = 1, b = 100) {
      (a - x[1])^2 + b * (x[2] - x[1]^2)^2
    }
  \end{lstlisting}
\end{frame}

\begin{frame}[c,fragile]
  \frametitle{Workflow}
  We have created a first R file, but how do we use it? Two major options:

  \begin{columns}[T]
    \begin{column}{0.45\textwidth}
      \begin{block}{\lstinline{devtools::install()}}
        Installs the package, like calling \lstinline{install.packages()}.

        \medskip\pause

        Robust but slow. Need to call \lstinline{library(rosenbrock)}
        to load package\footnote{Done automatically in R Studio}.
      \end{block}
    \end{column}

    \pause

    \begin{column}{0.45\textwidth}
      \begin{block}{\lstinline{devtools::load_all()}}
        Sources all of your code.

        \medskip

        Quick but not as robust.

      \end{block}
    \end{column}
  \end{columns}

  \pause\bigskip

  \begin{block}{Try It}
    Try both options and see if you can call your newly defined function, \lstinline{objective()}.
  \end{block}
\end{frame}

\begin{frame}[c]
  \begin{figure}[htpb]
    \centering
    \frame{\includegraphics[width=0.92\textwidth]{images/install-load-states.png}}
    \caption{%
      The various states of a package and how to move between them.
    }
  \end{figure}
\end{frame}

\begin{frame}[c,fragile]
  \frametitle{Exporting Functions}

  If you called \lstinline{devtools::load_all()} then everything is sourced and you can
  just call \lstinline{objective()} directly.

  \bigskip

  But if you use \lstinline{devtools::install()} and \lstinline{library(rosenbrock)}, the
  you would need to use \lstinline{rosenbrock:::objective()}. The reason is that
  the function is not yet exported.

  \medskip\pause

  \begin{block}{\texttt{NAMESPACE}}

    Decides what functions you want exported. But right now it just contains a comment:
    \begin{lstlisting}
      # Generated by roxygen2: do not edit by hand
    \end{lstlisting}

    \pause\medskip

    If you want to just export everything, you can remove this file and recreate it with this content:
    \begin{lstlisting}
      exportPattern("^[[:alpha:]]+")
    \end{lstlisting}

  \end{block}
\end{frame}

\begin{frame}[c,fragile]
  \frametitle{roxygen2}

  \textbf{roxygen2} is a package that helps with package documentation\footnote{More on this later.},
  but it can also be used for handling the namespace.

  \bigskip\pause

  To export a function, you need to place a special roxygen2 comment just before the function:
  \begin{lstlisting}
    #' @export
  \end{lstlisting}

  \medskip\pause

  Go ahead and place this before your \lstinline{objective()} definition. Then run
  \lstinline{devtools::document()} to roxygenize your package.

  \bigskip\pause

  Now \texttt{NAMESPACE} will (should) contain this:
  \begin{lstlisting}
    export(objective)
  \end{lstlisting}

  \medskip\pause

  Reinstall the package and see if you can call \lstinline{objective()} after
  loading it.
\end{frame}

\begin{frame}[c,fragile]
  \frametitle{Tests}

  \begin{block}{testthat}
    \begin{itemize}
      \item We have already encountered \textbf{testthat} for writing tests in a formalized way.
      \item But \textbf{testthat} was actually written especially for packages.
    \end{itemize}
  \end{block}

  \pause\bigskip

  Let's start using \textbf{testthat} with our package:
  \begin{lstlisting}
    usethis::use_testthat()
  \end{lstlisting}

  \bigskip\pause

  This creates some new files and directories:

  \medskip

  \dirtree{%
    .1 rosenbrock/.
    .2 tests/.
    .3 testthat/.
    .4 test-<some\_fun>.R\DTcomment{Your test file for \texttt{some\_fun()}}.
    .3 testthat.R.
  }
\end{frame}

\begin{frame}[c,fragile]
  \frametitle{A First Simple Test}

  For the Rosenbrock function, \(f^* = f(a,a^2) = f(1,1) = 0\). Let's make sure this is the case for us too!

  \bigskip\pause

  To create a test, we can use \lstinline[basicstyle=\ttfamily]{usethis::use_test()}.

  \bigskip\pause

  Call \lstinline[basicstyle=\ttfamily]{use_test("objective")}\footnote<3->{It's good practice to name the test file the same as the file where the function you're testing is defined.} and insert this:
  \begin{lstlisting}
    test_that("multiplication works", {
      # add a test using expect_equal()
    })
  \end{lstlisting}

  \medskip\pause

  \begin{block}{Check That Everything Works}
    Run \lstinline[basicstyle=\ttfamily]{devtools::test()}, and hopefully see:

    \medskip

    \begin{lstlisting}
      [ FAIL 0 | WARN 0 | SKIP 0 | PASS 1 ]
    \end{lstlisting}
  \end{block}
\end{frame}

\begin{frame}[c]
  \frametitle{Checking}
  \begin{block}{\texttt{R CMD check}}
    R contains functionality for checking that your package is built correctly and you can
    access this functionality through \lstinline{devtools::check()}.
  \end{block}

  \bigskip\pause

  No requirement that your package needs to pass these checks (if you're using it as a project), but it's
  good practice to make sure it does.

  \bigskip\pause

  \begin{description}[<+->]
    \item[ERROR] Major problem with your package
    \item[WARNING] Something that is most likely not great but not critical
    \item[NOTE] Typically small issues with your package
  \end{description}

  \bigskip\pause

  Now run \lstinline{devtools::check()}. Is there a problem? Yes, let's fix it!

\end{frame}

\begin{frame}[c,fragile]
  \frametitle{Metadata}

  The metadata for your package lives in \texttt{DESCRIPTION}. Right now it looks like this:
  \begin{lstlisting}[basicstyle=\ttfamily\small]
    Package: rosenbrock
    Title: What the Package Does (One Line, Title Case)
    Version: 0.0.0.9000
    Authors@R:
        person("First", "Last", , "first.last@example.com", role = c("aut", "cre"),
               comment = c(ORCID = "YOUR-ORCID-ID"))
    Description: What the package does (one paragraph).
    License: `use_mit_license()`, `use_gpl3_license()` or friends to pick a
        license
    Encoding: UTF-8
    Roxygen: list(markdown = TRUE)
    RoxygenNote: 7.3.2
  \end{lstlisting}

  \pause\bigskip

  For now we'll leave most of these files alone, but let's
  fix one thing: the license
\end{frame}

\begin{frame}[c,fragile]
  \frametitle{Licensing}

  \begin{block}{Why Do You Need a License?}
    \begin{itemize}[<+->]
      \item Licensing software tells other people about how they are allowed to reuse your
            code.
      \item If you do not provide a license, this generally means that \alert{nobody is allowed to copy,
              distribute, or modify your code.}

      \item If you have other contributors, then ``nobody'' includes \alert{you too!}
    \end{itemize}

  \end{block}

  \pause

  \begin{block}{Choosing a License}
    So we need to pick a license: for now we'll pick the MIT license.\footnote<5->{Read more about picking a license at \url{https://choosealicense.com}.}

    \medskip\pause

    \begin{lstlisting}
      usethis::use_mit_license()
    \end{lstlisting}

    \medskip

    This will add new files to your package: \texttt{LICENSE}, \texttt{LICENSE.md}, and
    modify \texttt{DESCRIPTION}, in which you should see:

    \begin{lstlisting}
      License: MIT + file LICENSE
    \end{lstlisting}
  \end{block}
\end{frame}

\begin{frame}[c,fragile]
  \frametitle{Dependencies}

  In R packages, you make dependencies explicit, defined in \texttt{DESCRIPTION}

  \medskip\pause

  \begin{block}{Gradient}
    Let's say that we want to compute the gradient for the Rosenbrock function.

    \medskip\pause

    One way to do so is to use numerical differentiation through the \textbf{numDeriv} package:

    \begin{lstlisting}
      gradient <- function(x, a = 1, b = 100) {
        numDeriv::grad(objective, x, a = a, b = b)
      }
    \end{lstlisting}
  \end{block}

  Now our package depends on \textbf{numDeriv}, so we need to add it
  to \texttt{DESCRIPTION}:
  \begin{lstlisting}
    usethis::use_package("numDeriv")
  \end{lstlisting}

  \medskip\pause

  In \texttt{DESCRIPTION}, you should now see this:

  \begin{lstlisting}
    Imports: 
      numDeriv
  \end{lstlisting}

\end{frame}

\begin{frame}[c,fragile]
  \frametitle{Rcpp}

  \begin{columns}[T]
    \begin{column}{0.48\textwidth}

      Rcpp works best in a package:
      \begin{itemize}
        \item No more manual sourcing (no need to call \lstinline{Rcpp::sourceCpp()})
        \item You don't need to add directives for dependencies to \textbf{RcppArmadillo} and other
              packages.
      \end{itemize}

      \pause\medskip

      We will rely on \textbf{roxygen2}. First, call
      \begin{lstlisting}
        usethis::use_package_doc()
      \end{lstlisting}
      to set up a package doc file in \texttt{R/rosenbrock-package.R}.
    \end{column}

    \pause

    \begin{column}{0.42\textwidth}
      Then use \lstinline{usethis::use_rcpp()} to put the pieces in place:

      \medskip

      \dirtree{%
        .1 rosenbrock/.
        .2 src/.
        .3 slop-package.cpp/.
      }

      \medskip\pause

      Now just need to run \texttt{devtools::document()} and
      \lstinline{devtools::load_all()} or \lstinline{devtools::install()}
      and no your code is available (but not exported).

      \medskip\pause

      To export, easiest is to write an R wrapper.
    \end{column}
  \end{columns}

\end{frame}

\begin{frame}[c]
  \frametitle{Documentation}

  Writing documentation is useful for others who want to use your code as well as for your future self.

  \medskip\pause

  Many aspects of documentation
  \begin{itemize}
    \item Comments in code
    \item Manual (help files)
    \item Long-form articles (vignettes)
  \end{itemize}

  \medskip\pause

  For the manual part, you can use \textbf{roxygen2} (it's main purpose).

\end{frame}

\begin{frame}[c]
  \begin{figure}[htpb]
    \centering
    \frame{\includegraphics[width=0.8\textwidth]{images/pkgs-workflow.png}}
    \caption{%
      The whole game
    }
  \end{figure}
\end{frame}

\begin{frame}[c]
  \frametitle{Exercise: Optimize the Rosenbrock Function}

  Write a gradient descent (or stochastic gradient descent) that
  minimizes the rosenbrock function.

  \medskip\pause

  Write the code in Rcpp. If you want, you can first write it in R to
  see that everything is working, and then port it.

  \medskip\pause

  Feel free to use generative AI to write the code.

  \medskip\pause

  Export everything and document the package.
\end{frame}

\begin{frame}[c]
  \frametitle{What We Didn't Cover}

  \begin{itemize}[<+->]
    \item Version control through git and github
    \item Metadata
    \item Publishing to CRAN
  \end{itemize}

\end{frame}

\section{Oral Examination Prep}

\begin{frame}[c]
  \frametitle{The Five Points}

  Remember the five points:
  \begin{itemize}
    \item How can you test that your implementation is correct?
    \item Can you implement alternative solutions?
    \item Can the code be restructured e.g. by modularization, abstraction or object oriented programming to improve generality, extendability and readability?
    \item How does the implementation perform (benchmarking)?
    \item Where are the bottlenecks (profiling), and what can you do about them?
  \end{itemize}


\end{frame}

\section{Course Summary}

\begin{frame}[c]
  \frametitle{Course Summary}

  \begin{block}{Statistical Topics}
    \begin{description}[<+->][Optimization]
      \item[Smoothing] Kernel density smoothing and splines (topic 1)
      \item[Simulation] MC methods: rejection and importance sampling (topic 2)
      \item[Optimization] The EM algorithm (topic 3), gradient descent and stochastic optimization (topic 4)
    \end{description}
  \end{block}

  \pause

  \begin{block}{Computational Topics}
    \begin{itemize}
      \item Debugging
      \item Profiling
      \item Benchmarking
      \item Debugging
      \item Writing performant code
    \end{itemize}
  \end{block}
\end{frame}

\appendix

% \begin{frame}[allowframebreaks]{References}
%   \printbibliography[heading=none]
% \end{frame}


\end{document}

